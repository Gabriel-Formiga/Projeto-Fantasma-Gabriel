% Options for packages loaded elsewhere
% Options for packages loaded elsewhere
\PassOptionsToPackage{unicode}{hyperref}
\PassOptionsToPackage{hyphens}{url}
\PassOptionsToPackage{dvipsnames,svgnames,x11names}{xcolor}
%
\documentclass[
  portuguese,
]{estat/estat}
\usepackage{xcolor}
\usepackage[left=3cm,right=2cm,top=3cm,bottom=2cm]{geometry}
\usepackage{amsmath,amssymb}
\setcounter{secnumdepth}{5}
\usepackage{iftex}
\ifPDFTeX
  \usepackage[T1]{fontenc}
  \usepackage[utf8]{inputenc}
  \usepackage{textcomp} % provide euro and other symbols
\else % if luatex or xetex
  \usepackage{unicode-math} % this also loads fontspec
  \defaultfontfeatures{Scale=MatchLowercase}
  \defaultfontfeatures[\rmfamily]{Ligatures=TeX,Scale=1}
\fi
\usepackage{lmodern}
\ifPDFTeX\else
  % xetex/luatex font selection
  \setmainfont[]{Arial}
\fi
% Use upquote if available, for straight quotes in verbatim environments
\IfFileExists{upquote.sty}{\usepackage{upquote}}{}
\IfFileExists{microtype.sty}{% use microtype if available
  \usepackage[]{microtype}
  \UseMicrotypeSet[protrusion]{basicmath} % disable protrusion for tt fonts
}{}
\makeatletter
\@ifundefined{KOMAClassName}{% if non-KOMA class
  \IfFileExists{parskip.sty}{%
    \usepackage{parskip}
  }{% else
    \setlength{\parindent}{0pt}
    \setlength{\parskip}{6pt plus 2pt minus 1pt}}
}{% if KOMA class
  \KOMAoptions{parskip=half}}
\makeatother
% Make \paragraph and \subparagraph free-standing
\makeatletter
\ifx\paragraph\undefined\else
  \let\oldparagraph\paragraph
  \renewcommand{\paragraph}{
    \@ifstar
      \xxxParagraphStar
      \xxxParagraphNoStar
  }
  \newcommand{\xxxParagraphStar}[1]{\oldparagraph*{#1}\mbox{}}
  \newcommand{\xxxParagraphNoStar}[1]{\oldparagraph{#1}\mbox{}}
\fi
\ifx\subparagraph\undefined\else
  \let\oldsubparagraph\subparagraph
  \renewcommand{\subparagraph}{
    \@ifstar
      \xxxSubParagraphStar
      \xxxSubParagraphNoStar
  }
  \newcommand{\xxxSubParagraphStar}[1]{\oldsubparagraph*{#1}\mbox{}}
  \newcommand{\xxxSubParagraphNoStar}[1]{\oldsubparagraph{#1}\mbox{}}
\fi
\makeatother


\usepackage{longtable,booktabs,array}
\usepackage{calc} % for calculating minipage widths
% Correct order of tables after \paragraph or \subparagraph
\usepackage{etoolbox}
\makeatletter
\patchcmd\longtable{\par}{\if@noskipsec\mbox{}\fi\par}{}{}
\makeatother
% Allow footnotes in longtable head/foot
\IfFileExists{footnotehyper.sty}{\usepackage{footnotehyper}}{\usepackage{footnote}}
\makesavenoteenv{longtable}
\usepackage{graphicx}
\makeatletter
\newsavebox\pandoc@box
\newcommand*\pandocbounded[1]{% scales image to fit in text height/width
  \sbox\pandoc@box{#1}%
  \Gscale@div\@tempa{\textheight}{\dimexpr\ht\pandoc@box+\dp\pandoc@box\relax}%
  \Gscale@div\@tempb{\linewidth}{\wd\pandoc@box}%
  \ifdim\@tempb\p@<\@tempa\p@\let\@tempa\@tempb\fi% select the smaller of both
  \ifdim\@tempa\p@<\p@\scalebox{\@tempa}{\usebox\pandoc@box}%
  \else\usebox{\pandoc@box}%
  \fi%
}
% Set default figure placement to htbp
\def\fps@figure{htbp}
\makeatother



\ifLuaTeX
\usepackage[bidi=basic]{babel}
\else
\usepackage[bidi=default]{babel}
\fi
\ifPDFTeX
\else
\babelfont{rm}[]{Arial}
\fi
% get rid of language-specific shorthands (see #6817):
\let\LanguageShortHands\languageshorthands
\def\languageshorthands#1{}


\setlength{\emergencystretch}{3em} % prevent overfull lines

\providecommand{\tightlist}{%
  \setlength{\itemsep}{0pt}\setlength{\parskip}{0pt}}



 


\authors{%
    Estatiano 1 \\
    Estatiano 2\\
    Estatiano 3\\
}

% escreva o nome do cliente aqui
% se for mais de um separe por \\
\client{%
    ESTAT
}
% Baixando pacotes
\RequirePackage{fancyhdr}
\RequirePackage{graphicx}

\setlength\headheight{28pt}  

\setlength{\parindent}{15pt} % Adiciona indentação nos parágrafos
\setlength{\parskip}{0pt} % Adiciona 0 espaço entro os parágrafos

\let\oldsection\section
\renewcommand\section{\clearpage\oldsection}
\makeatletter
\@ifpackageloaded{float}{}{\usepackage{float}}
\floatstyle{plain}
\@ifundefined{c@chapter}{\newfloat{quadro}{h}{loquad}}{\newfloat{quadro}{h}{loquad}[chapter]}
\floatname{quadro}{Quadro}
\floatstyle{plaintop}
\restylefloat{quadro}
\newcommand*\listofquadros{\listof{quadro}{List of Testes}}
\makeatother
\makeatletter
\@ifpackageloaded{caption}{}{\usepackage{caption}}
\AtBeginDocument{%
\ifdefined\contentsname
  \renewcommand*\contentsname{Índice}
\else
  \newcommand\contentsname{Índice}
\fi
\ifdefined\listfigurename
  \renewcommand*\listfigurename{Lista de Figuras}
\else
  \newcommand\listfigurename{Lista de Figuras}
\fi
\ifdefined\listtablename
  \renewcommand*\listtablename{Lista de Tabelas}
\else
  \newcommand\listtablename{Lista de Tabelas}
\fi
\ifdefined\figurename
  \renewcommand*\figurename{Figura}
\else
  \newcommand\figurename{Figura}
\fi
\ifdefined\tablename
  \renewcommand*\tablename{Tabela}
\else
  \newcommand\tablename{Tabela}
\fi
}
\@ifpackageloaded{float}{}{\usepackage{float}}
\floatstyle{ruled}
\@ifundefined{c@chapter}{\newfloat{codelisting}{h}{lop}}{\newfloat{codelisting}{h}{lop}[chapter]}
\floatname{codelisting}{Listagem}
\newcommand*\listoflistings{\listof{codelisting}{Lista de Listagens}}
\captionsetup{labelsep=colon}
\makeatother
\makeatletter
\makeatother
\makeatletter
\@ifpackageloaded{caption}{}{\usepackage{caption}}
\@ifpackageloaded{subcaption}{}{\usepackage{subcaption}}
\makeatother
\usepackage{bookmark}
\IfFileExists{xurl.sty}{\usepackage{xurl}}{} % add URL line breaks if available
\urlstyle{same}
\hypersetup{
  pdflang={pt},
  colorlinks=true,
  linkcolor={black},
  filecolor={black},
  citecolor={black},
  urlcolor={black},
  pdfcreator={LaTeX via pandoc}}


\author{}
\date{}
\begin{document}

% Limpando tudo
\fancyhf{} 

% Ajustes do header
\fancyhead[L]{} % limpando o lado esquerdo
\fancyhead[R]{\includegraphics[width=0.20\textwidth]{estat/imagens/estat.png}} % adicionando logo no canto direito
\renewcommand{\headrulewidth}{0pt}   % sem linha embaixo da logo

% Ajustes de fim de página
\fancyfoot[R]{\textcolor{white}{\thepage}} % Número em branco no canto direito

% Aplicando o estilo que acabamos de criar
\pagestyle{fancy} 


\labelformat{quadro}{\textbf{#1}}

\renewcommand*\contentsname{Sumário}
{
\hypersetup{linkcolor=}
\setcounter{tocdepth}{3}
\tableofcontents
}

\section{Análises}\label{anuxe1lises}

\subsection{Análise 3}\label{anuxe1lise-3}

Esta análise, busca-se compreender e identificar as caracteristicas
etarais do clientes de diferentes loja em Âmabr seco, com o objetivo de
descobrir se existe uma variação significativa no perfil das idades dos
clientes entre as lojas da região. Ná analise será utilizado as
variaveis idade, que é quantitativa discreta e loja, que é qualitativa
nominal. A visualização será através de um boxplot bivariado, o qual é
mais adequado para essa analise, onde cada boxplot será representado por
uma loja diferente da região, a linha central nele representara a
mediana das iades e o losangolo representa a media, e por um quadro,
onde terá os dados das principais medidas calculadas na análise

\begin{figure}[H]

\caption{\label{fig-idade_loja}Boxplot bivariado da idade do cliente por
Nome da loja em Ambar Seco}

\centering{

\pandocbounded{\includegraphics[keepaspectratio]{entrega_3_files/figure-pdf/fig-idade_loja-1.pdf}}

}

\end{figure}%

\begin{quadro}[H]

\caption{\label{quad-idade_loja}Através do gráfico
\(\ref{fig-idade_loja}\) e pelo quadro {[}**Quadro** @quad-{]} é
possivel visualizar uma notavel diferença no perfil etario da loja
Vendinha Rápida com as outras três lojas da região. A loja Ferraria Seca
possui a clientela mais jovem da região, tendo uma média de 33,67 anos e
mediana de 30,5. As lojas Saloon e Banco Careca possuem uma clientela
mais jovem, com as medianas de 32,5 e 34 anos, tendo um perfil etario
bem proximo com a de Ferraria Seca. A loja que tem um publico mais
velho, e mais distinto das demais, é a Vendinha Rápida, que possui uma
media de 40,34 e uma mediana de 41 anos. O Banco careca e a Vendinha
Rápida possuem uma menor dispersão de idade, por ter baixos desvios
padrões, sendo 5,57 e 6,03, ou seja, as clientela dessas lojas possui
uma idade bastante semelhante. Mas a Vendinha Rápida é a que possui a
menor dispersão entre todas e o Banco Careca possui a maior homogenidade
de clientes, tendo um intervalo interquartil de 9 anos. Já as lojas
Ferraria Seca e Saloon possuem uma maior dispersão entre as lojas, por
ter altos desvios padrões, sendo 13,31 e 12,69. Com isso é possivel
perceber que nessas duas lojas possuem uma clientela mais jovem, mas
tendo uma grande mistura de idades, o que é possivel visualizar pelos
outliers, que são valores discrepantes em relação as demais idades, A
Ferraria Seca e Saloon possuem clientes com idades mais elevada, mesmo
tendo clientes mais jovens, fazendo que a media desas lojas aumentem em
relação a mediana. As duas lojas também possuem uma assimetria mais a
direita, sendo positiva, ressaltando que existe uma grande diferença nas
idades da clientela das lojas. Em contra partida, O Banco careca possui
a distribuição mais simetrica em relação as outras, tendo a menor
diferença de idade dos clientes, e ela não apresenta valores
discrepantes. A Vendinha Rápida possui uma assimetria mais a esquerda,
sendo negativa, mostrando que a maioria dos clientes são mais velho,
tendo poucos clientes jovens, mas também não possui valores
discrepantes, confirmando que não existe um grande diferença de idade
dos clientes.}

\centering{

}

\end{quadro}%




\end{document}
